% ptex2pdf -l -synctex=1 main.tex
\documentclass{jsarticle}
\usepackage{amsmath}
\usepackage{amssymb}
\usepackage{amsthm}
\usepackage{amscd}
\usepackage{xypic}
\usepackage{ascmac}
\usepackage[hyphens]{url}
\usepackage{braket}
\usepackage[dvipdfmx,hiresbb,final]{graphicx}
\newcommand{\CC}{\mathbb{C}}
\newcommand{\RR}{\mathbb{R}}
\newcommand{\QQ}{\mathbb{Q}}
\newcommand{\ZZ}{\mathbb{Z}}
\newcommand{\NN}{\mathbb{N}}
\newcommand{\FF}{\mathbb{F}}
\newcommand{\PP}{\mathbb{P}}
\newcommand{\GG}{\mathbb{G}}
\newcommand{\PhiT}{\trans{\Phi}}
\newcommand{\calN}{{\cal N}}
\newcommand{\calD}{{\cal D}}
\newcommand{\calL}{{\cal L}}
\newcommand{\calQ}{{\cal Q}}
\newcommand{\calW}{{\cal W}}
\newcommand{\half}{\frac{1}{2}}
\newcommand{\ignore}[1]{}
\newcommand{\con}{\nabla}
\newcommand{\dcon}{\con^{*}}
\newcommand{\mcon}{\overline{\con}}
\newcommand{\cris}[3]{\Gamma_{#1 #2}^{#3}}
\newcommand{\marud}[1]{\partial_{#1}}
\newcommand{\marudd}[1]{\frac{\partial}{\partial #1}}

\newcommand{\makeop}[1]{\mathop{\mathrm{#1}}\nolimits}
\newcommand{\supp}{\makeop{supp}}

\theoremstyle{definition}
\newtheorem{theorem}{定理}
\newtheorem*{theorem*}{定理}
\newtheorem{lemma}{補題}
\newtheorem*{lemma*}{補題}
\newtheorem{definition}[theorem]{定義}
\newtheorem*{definition*}{定義}

\newcommand{\mydescription}[1]{
\begin{description}
\setlength{\itemindent}{2zw}
\setlength{\leftskip}{-2zw}
\setlength{\labelsep}{1zw}
#1
\end{description}
}

\numberwithin{theorem}{section}

\begin{document}
『情報幾何学の基礎』(藤原彰夫)輪読会5章資料 by @herumi

\section{復習}
$M$を$C^\infty$多様体, $\chi(M)$を$M$のベクトル場全体とする.

多様体$M$, $N$に対して$f:M \rightarrow N$を考える。
$f_*:TM \rightarrow TN$を$X \in TM$, $g:N \rightarrow \RR$に対して
\[
f_*(X)(g):=X(f^*g)
\]
と定義する. $f_*$を$df$と書くこともある.

多様体$S$と写像$g:N \rightarrow S$があると
\[
TM \xrightarrow{f_*} TN \xrightarrow{g_*} TS
\]
が誘導される.
\[
g_* \circ f_* = (g \circ f)_*.
\]

$\RR^n_+:=\Set{(x_1, \dots, x_n)|x_i>0}$,
$S_{n-1}:=\Set{(x_1, \dots, x_n)|x_i>0, \sum_i x_i = 1}$とする.

$S_{n-1}$に$\RR^n_+$の相対位相を入れる(開集合が$U \cap \RR^n_+$の形).
\[
\varphi:S_{n-1} \ni (x_1, \dots, x_n) \mapsto (x_1, \dots, x_{n-1}) \in \RR^{n-1}_+
\]
$T^{n-1}:=\Set{(x_1, \dots, x_n)|x_i>0, \sum_i x_i < 1}$
とすると$\varphi:S_{n-1} \rightarrow T^{n-1}$はhomeo.

$n$行$l$列の行列$Q=(Q_{ij})$, $Q_{ij}>0$, $\sum_j Q_{ij}=1$をとる($1 \le i \le n$, $1 \le j \le l$). 縦に$n$個, 横に$l$個.
\[
f:S_{n-1} \ni (x_i) \mapsto (\sum_i x_i Q_{ij}) \in S_{l-1}
\]
とする.

$|x|:=\sum_i x_i$とすると
\[
|f(x)|=\sum_j (\sum_i x_i Q_{ij})=\sum_i x_i (\sum_j Q_{ij})=\sum_i x_i =|x|.
\]

特に$|x|=1$なら$|f(x)|=1$なので$f$を$S_{n-1}$に制限した写像$f$について$f:S_{n-1}\rightarrow S_{l-1}$がwell-defined.

$h_n:\RR^n_+ \rightarrow S_{n-1}$を$h_n(x):=(1/|x|)x$とする.

$h_n$を$S_{n-1}$に制限すると$h_n=id_{S_{n-1}}$.

また
\[
f(h_n(x))=f(x/|x|)=(1/|x|)f(x)=(1/|f(x)|)f(x)=h_l(f(x)).
\]
つまり次が可換.
\[
\xymatrix{
\RR^n_+ \ar[r]^-f \ar[d]_{h_n} & \RR^l_+ \ar[d]^-{h_l} \\
S_{n-1} \ar[r]_-f & S_{l-1} \ar@{}[lu]|{\circlearrowright}
}
\]

\begin{lemma}
$g^{[n]}:TS_{n-1} \times TS_{n-1} \rightarrow \RR$について
\[
g^{[n]}(X,Y)=g^{[l]}(f_* X, f_* Y) \quad \text{ for } X, Y \in TS_{n-1}
\]
が成り立つとする.

このとき$\overline{g}^{[n]} : T\RR^n_+ \times T\RR^n_+ \rightarrow \RR$を
\[
\overline{g}^{[n]}(\overline{X}, \overline{Y}):=g^{[n]}(h_{n*} \overline{X}, h_{n*} \overline{Y}) \quad \text{ for } \overline{X}, \overline{Y} \in T\RR^n_+
\]
とすると$\overline{g}$は$g$の拡張となっている.
\end{lemma}
(証明)

$\iota:S_{n-1}\rightarrow \RR^n_+$をinclusionとすると$x \in S_{n-1}$について$h_n(\iota(x))=x$.
よって
\[
\overline{g}^{[n]}(\iota_* X, \iota_* Y)=g^{[n]}(h_{n*}  \iota_* X, h_{n*} \iota_* Y)=g^{[n]}(X,Y).
\]

また
\begin{eqnarray*}
\overline{g}^{[l]}(f_* \overline{X}, f_* \overline{Y})&=&g^{[l]}(h_{l*} f_* \overline{X}, h_{l*} f_* \overline{Y})=g^{[l]}(f_* h_{n*} \overline{X}, f_* h_{n*} \overline{Y})\\
&=&g^{[n]}(h_{n*} \overline{X}, h_{n*} \overline{Y})=\overline{g}^{[n]}(\overline{X}, \overline{Y}).
\end{eqnarray*}

\end{document}
